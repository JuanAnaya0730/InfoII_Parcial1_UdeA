\documentclass{article}
\usepackage[utf8]{inputenc}
\usepackage[spanish]{babel}
\usepackage{listings}
\usepackage{graphicx}
\graphicspath{ {imagenes/} }
\usepackage{cite}

\begin{document}

\begin{titlepage}
    \begin{center}
        \vspace*{1cm}
            
        \Huge
        \textbf{Informatica II}
            
        \vspace{0.5cm}
        \LARGE
        Parcial I
            
        \vspace{1.5cm}
            
        \textbf{Juan Sebastian Anaya Regino}
        
        \vspace{0.9cm}
        \centering
        \includegraphics[width=6cm]{imagenes/logo.png}
            
        \vfill
            
        \vspace{0.8cm}
            
        \Large
        Despartamento de Ingeniería Electrónica y Telecomunicaciones\\
        Universidad de Antioquia\\
        Medellín\\
        Marzo de 2021
            
    \end{center}
\end{titlepage}

\tableofcontents
\newpage

\section{Análisis}
    Para enfrentarse a cualquier problema es necesario realizar un analisis exhaustivo del mismo, con el fin de llegar a la mayor conprension posible que sera de mucha utilidad a la hora de plantear una posible solucion. Si bien el ploblema planteado es aparentemente secillo no es muy prudente lanzarse a una solucion sin antes analizar y tener en cuenta todos y cada uno de los requisitos con los que hay que cumplir para presentar una solucion.
    
    De entrada al problema vemos que lo primero a lo que nos debemos enfrentar es al ensamblaje de un circuito electronico para controlar una matriz de sesenta y cuatro leds(64), aunque ciertamente no se esta muy familiarizado con el diseño de circuitos es imposible no imaginarse mil y una forma de armar el ese cicuito, pero sigamos analizando a ver que limitantes nos depara el reto. Para no estancarse
    en el diseño del circuito supongamos que ya lo solucionamos, entonces al avanzar un poco en el problema nos encontramos con que se debe crear una funcion de nombre especifico que permita a un usuario exterior ver el correcto funcionaminto de todos los leds encendiendolos al mismo tiempo como se muestra en la figura{\ref{fig:matrizllena}}.
    
    \begin{figure}[h]
    \includegraphics[width=4cm]{MatrizLlena.png}
    \centering
    \caption{Matriz totalmente encendida}
    \label{fig:matrizllena}
    \end{figure}
        
\section{Plan de actividades}


\end{document}
